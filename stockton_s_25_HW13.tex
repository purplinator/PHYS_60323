%%%%%%%%%%%%%%%%%%%%%%%%%%%%%%%%%%%%%%%%%%%%%%%%%%%%%%%%%%%%
%%%%%%%%%%%%%%%%%%%%%%%%%%%%%%%%%%%%%%%%%%%%%%%%%%%%%%%%%%%%
%%%%%%%%%%%%%%%%%%%%%%%%%%%%%%%%%%%%%%%%%%%%%%%%%%%%%%%%%%%%
%%%%%%%%%%%%%%%%%%%%%%%%%%%%%%%%%%%%%%%%%%%%%%%%%%%%%%%%%%%%
%%%%%%%%%%%%%%%%%%%%%%%%%%%%%%%%%%%%%%%%%%%%%%%%%%%%%%%%%%%%
\documentclass[12pt]{article}
\usepackage{fancyhdr}
\usepackage{pslatex}
\usepackage{epsfig}
\usepackage{times}
\usepackage{amsmath}
\usepackage{mathrsfs}
\usepackage[dvipsnames]{xcolor}
\usepackage[hidelinks]{hyperref}%renewcommand{\topfraction}{1.0}
\renewcommand{\topfraction}{1.0}
\renewcommand{\bottomfraction}{1.0}
\renewcommand{\textfraction}{0.0}
\setlength {\textwidth}{6.6in}
\hoffset=-1.0in
\oddsidemargin=1.00in
\marginparsep=0.0in
\marginparwidth=0.0in                                                                               
\setlength {\textheight}{9.0in}
\voffset=-1.00in
\topmargin=1.0in
\headheight=0.0in
\headsep=0.00in
\footskip=0.50in                                         
\setcounter{page}{1}
\begin{document}
\def\pos{\medskip\quad}
\def\subpos{\smallskip \qquad}
\newfont{\nice}{cmr12 scaled 1250}
\newfont{\name}{cmr12 scaled 1080}
\newfont{\swell}{cmbx12 scaled 800}
%%%%%%%%%%%%%%%%%%%%%%%%%%%%%%%%%%%%%%%%%%%%%%%%%%%%%%%%%%%%
%     DO NOT CHANGE ANYTHING ABOVE THIS LINE
%%%%%%%%%%%%%%%%%%%%%%%%%%%%%%%%%%%%%%%%%%%%%%%%%%%%%%%%%%%%
%     DO NOT CHANGE ANYTHING ABOVE THIS LINE
%%%%%%%%%%%%%%%%%%%%%%%%%%%%%%%%%%%%%%%%%%%%%%%%%%%%%%%%%%%%
%     DO NOT CHANGE ANYTHING ABOVE THIS LINE
%%%%%%%%%%%%%%%%%%%%%%%%%%%%%%%%%%%%%%%%%%%%%%%%%%%%%%%%%%%%

\begin{center}
{\Large
PHYS 20323/60323: Fall 2025 - LaTeX Example
}\\
%%%%%%%%%%%%%%%%%%%%%%%%%%%%%%%%%%%%%%%%%%%%%%%%%%%%%%%%%%%%
% {\large Project: Your Name Here}\\\vskip0.25in
%%%%%%%%%%%%%%%%%%%%%%%%%%%%%%%%%%%%%%%%%%%%%%%%%%%%%%%%%%%%
\end{center}
%%%%%%%%%%%%%%%%%%%%%%%%%%%%%%%%%%%%%%%%%%%%%%%%%%%%%%%%%%%%
% Section Heading
%%%%%%%%%%%%%%%%%%%%%%%%%%%%%%%%%%%%%%%%%%%%%%%%%%%%%%%%%%%%
% \noindent {\bf PROJECT INFORMATION:} \\

% Write your text here

%%%%%%%%%%%%%%%%%%%%%%%%%%%%%%%%%%%%%%%%%%%%%%%%%%%%%%%%%%%%
% Section Heading
%%%%%%%%%%%%%%%%%%%%%%%%%%%%%%%%%%%%%%%%%%%%%%%%%%%%%%%%%%%%
% \vskip0.1in
% \noindent {\bf PURPOSE:} \\

% Write your text here

% You can use superscripts (m s$^{-1}$) or subscripts (M$_{*}$). 
\begin{enumerate}
    \item At time t = 0 a particle is represented by the wave function

\begin{align*}
    \Psi(x)=\begin{cases}
        A{x\over a}, & 0\le x\le a\\
        A{(b-x)\over (b-a)}, & a\le x \le b\\
        0, & \text{otherwise}
    \end{cases}
\end{align*}
where $A$, $a$, and $b$, are constants.

\begin{itemize}
    \item[(a)] (3.3 points) Normalize $\Psi$ (i.e., find A terms of a and b).
    \item[(b)] (3.3 points) Where is the particle likely to be found at $t = 0$?.
    \item[(c)] (3.4 points) What is the expectation value of $x$?. 
\end{itemize}

\item \textbf{The following questions refer to stars in the Table below}.\\
\textit{Note: There may be multiple answers.}

\begin{center}
\begin{tabular}{|l|c|r|r|r|r|r|}\hline
Name & Mass & Luminosity & Lifetime & Temperature & Radius & Variable \\\hline
$\delta$ Scu.   & $2.0\; M_\odot$   &    & $5.0\times10^8$ years &  & $2.0\; R_\odot$ & Y \\ \hline
$\gamma$ Del.   & $0.7\; M_\odot$  & & $4.5\times 10^{10}$ years & $5000$ K & & N   \\ \hline
$\beta$ Cyg.   & $1.3\; M_\odot$   & $3.5\; L_\odot$ &  & & &Y   \\\hline
$\eta$ Car. & $60.\; M_\odot$& $10^6\; L_\odot$ & $8.0\times10^5$ years & & &Y   \\\hline
$\varepsilon$ Eri. & $6.0\; M_\odot$ & $10^3\; L_\odot$ & & 20,000 K & &N\\\hline
$\alpha$ Cen. & $1.0\; M_\odot$ & & &6000 K & $1.0\; R_\odot$ & N\\\hline
\end{tabular}\vskip 0.2in
\end{center}

\begin{itemize}
    \item[(a)] (4 points) Which of these stars will produce a planetary nebula.
    \item[(b)]  (4 points) Elements heavier than \texttt{Carbon} will be produced in which stars.
\end{itemize}

\item An electron is found to be in the spin state (in the $z$-basis): $\chi=A\begin{pmatrix}
    3i\\4
\end{pmatrix}$

\begin{itemize}
    \item[(a)] (5 points)  Determine the possible values of A such that the state is normalized.
    \item[(b)] (5 points) Find the expectation values of the operators {\color{red}$S_x$}, {\color{purple}$S_y$}, {\color{orange}$S_z$}, $\Vec{S}^2$.
\end{itemize}

The matrix representations in the $z$-basis for the components of electron spin operators are
given by:

\vspace{0.5cm}
$\color{red}{\textbf{S}_\textbf{x}={\hbar\over 2}\begin{pmatrix}
    0&1 \\ 1&0
\end{pmatrix};} \qquad\color{purple}{\textbf{S}_\textbf{y}={\hbar\over 2}\begin{pmatrix}
    0&-i \\ i&0
\end{pmatrix};} 
\qquad\color{orange}{\textbf{S}_\textbf{z}={\hbar\over 2}\begin{pmatrix}
    1&0 \\ 0&-1
\end{pmatrix}}$





\end{enumerate}

%%%%%%%%%%%%%%%%%%%%%%%%%%%%%%%%%%%%%%%%%%%%%%%%%%%%%%%%%%%%
% Bullet Point & Numbered list - lists can also be nested as below
%%%%%%%%%%%%%%%%%%%%%%%%%%%%%%%%%%%%%%%%%%%%%%%%%%%%%%%%%%%%


%%%%%%%%%%%%%%%%%%%%%%%%%%%%%%%%%%%%%%%%%%%%%%%%%%%%%%%%%%%%
% Section Heading
%%%%%%%%%%%%%%%%%%%%%%%%%%%%%%%%%%%%%%%%%%%%%%%%%%%%%%%%%%%%



%%%%%%%%%%%%%%%%%%%%%%%%%%%%%%%%%%%%%%%%%%%%%%%%%%%%%%%%%%%%

%%%%%%%%%%%%%%%%%%%%%%%%%%%%%%%%%%%%%%%%%%%%%%%%%%%%%%%%%%%%



%%%%%%%%%%%%%%%%%%%%%%%%%%%%%%%%%%%%%%%%%%%%%%%%%%%%%%%%%%%%
% Figures can be inserted
%%%%%%%%%%%%%%%%%%%%%%%%%%%%%%%%%%%%%%%%%%%%%%%%%%%%%%%%%%%%
% \begin{figure}[h!]
% \begin{center}
% \includegraphics{hydrogen.eps}
% \end{center}
% \caption{This is the energy levels for the Hydrogen atom}
% \end{figure}
% \begin{figure}[h]
% \begin{center}
% \includegraphics[scale=0.6,angle=90]{hydrogen.eps} 
% \end{center}
% \caption{This is the energy levels for the Hydrogen atom sideways
% \label{figure:map}}
% \end{figure}

% \clearpage
% \noindent {\bf ANALYSIS (and math):} \\

% And go into Math mode $A_x = \int_{n-1}^{n+10}  f(x) \mathrm{d}x$ 
% in the normal text or insert an equation:
% \begin{equation}
% 	F = ma \\
% 	   =m \frac{dv}{dt} 
% \end{equation}


% or if you want to go wild:
% \begin{equation}
% \prod_{j\ge0}\left( \sum_{k\ge0} a_{jk}z^k \right) = \sum_{n\ge0} z^n 
% \left( \sum_{k_0,k_1\ldots\ge0 \atop k_0+k_1\cdots=0} a_{0k_0} a_{1k_1}\ldots \right)
% \end{equation}
% and continue your text.


%%%%%%%%%%%%%%%%%%%%%%%%%%%%%%%%%%%%%%%%%%%%%%%%%%%%%%%%%%%%
% Section Heading
%%%%%%%%%%%%%%%%%%%%%%%%%%%%%%%%%%%%%%%%%%%%%%%%%%%%%%%%%%%%









%%%%%%%%%%%%%%%%%%%%%%%%%%%%%%%%%%%%%%%%%%%%%%%%%%%%%%%%%%%%
% Tables are created easily
%%%%%%%%%%%%%%%%%%%%%%%%%%%%%%%%%%%%%%%%%%%%%%%%%%%%%%%%%%%%
% \begin{center}
% \begin{tabular}{|l|crrrr|}\hline\hline
% Header 1 & Header 2 & Header 3 & temp & temp &\\\hline\hline
% Data 1a   & Data 2a   &  Data 3a  & temp & temp &  \\
% Data 1b   & Data 2b  &  Data 3b & temp & temp &   \\
% Data 1c   & Data 2c   &  Data 3c & temp & temp &   \\\hline
% \end{tabular}\vskip 0.2in
% \end{center}


%%%%%%%%%%%%%%%%%%%%%%%%%%%%%%%%%%%%%%%%%%%%%%%%%%%%%%%%%%%%
% Section Heading
%%%%%%%%%%%%%%%%%%%%%%%%%%%%%%%%%%%%%%%%%%%%%%%%%%%%%%%%%%%%
% \vskip0.1in
% \noindent {\bf CONCLUSION:}\\

% Write your text here.


%%%%%%%%%%%%%%%%%%%%%%%%%%%%%%%%%%%%%%%%%%%%%%%%%%%%%%%%%%%%



\end{document}
